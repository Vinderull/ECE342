\documentclass{article}
\usepackage{fullpage} % sets more standardized margins
\usepackage{graphicx} % some graphics functions I use 
\usepackage{abstract} % abstract function
\usepackage{amsmath}  % math stuff
\usepackage{float}
\usepackage{mathrsfs}
\usepackage[utf8]{inputenc}
\usepackage[document]{ragged2e}
\usepackage{subfiles}
\usepackage{caption}
\usepackage{subcaption}
\usepackage{verbatim}


\renewcommand{\absnamepos}{flushleft} % left justifies abstract
\setlength{\absleftindent}{0pt}
\setlength{\absrightindent}{0pt}
%\captionsetup[figure]{font=small,skip=0pt}


% % %
% Set up IEEE style paragraphing
% % %
\setlength{\parskip}{1em} % The \par command now skips a line between paragraphs, eliminates warnings from using the \par or \\ commands
\setlength{\parindent}{0em} % Left justifies paragraphs after a \par command 


\begin{document}

\pagenumbering{gobble} % Turn off page numbering for titles and tables


\begin{titlepage}
    \begin{center}
        
         \vspace*{1.5cm}
        
    %Berkay suggested this title in the meeting
         \textbf{{\Huge Task 4: Operational Amplifier}}
        
         \vspace{0.5cm}
         \textbf{Discrete Amplifier}
        
          \vspace{.5cm}
        
         \textbf{{\Large Joseph Arsenault \\ Ryan Dufour \\ Phil Robb \newline}}
    \vspace{1cm}
    
        
        
        \begin{abstract}
        
        
        
        \noindent   The design, simulation, and construction of experiments to measure the performance of an operation amplifier is explored. The operational amplifier consists of four discrete stages: a cascoded current mirror, an actively loaded differential pair, a common source amplifier, and finally a Bipolar Junction Transistor class B amplifier. The operational amplifier was designed to operate with a bias current of 400$\mu$A using the ALD1106 and ALD1107 series MOSFETS. The common mode rejection ratio was 75 dB, the common mode gain, Acm, for 100mV input was -10 dB, the Acm for 1 V input was -9 dB and differential gain, Ad, was found to be 67 dB.
        
        
        \end{abstract}
        
        
        \vspace{03cm}
        \textbf{
        Electrical and Computer Engineering\\
        University of Maine\\
        ECE - 343\\\ \today}
    \vspace{.5cm}
        \begin{figure}[H]
        \centering
        \includegraphics[scale = 0.1]{Images/barcode.png}
\end{figure}
    \end{center}
\end{titlepage}


\tableofcontents

\newpage

\newpage
\listoffigures
\newpage 
\clearpage

\pagenumbering{arabic} % Turn on page numbering



\section{Introduction}
\subfile{Introduction/Introduction}
  
 
  %\section{Circuit Development}
    %\subfile{CircuitDevelopment/Circuit.tex}
  
  \section{Circuit Development}
    \subfile{Simulations/simulations.tex}
    
  %\subfile{CircuitDevelopment/cd4007SIM.tex}

   % \subfile{CircuitDevelopment/NMOS.tex}
     
 
    
  \section{Experimental Implementation}
    \subfile{ExperimentalImplementation/experimental.tex}
    
  \section{Discussion}
     \subfile{Discussion/Discussion.tex}
    
    
    \section{Conclusion}
        \subfile{Conclusion/Conclusion.tex}

    



\clearpage

\appendix

\begin{thebibliography}{00}

\bibitem{b1} All About Circuits. (2018) Single-ended differential amplifiers [Online]: https://www.allaboutcircuits.com/textbook/semiconductors/chpt-8/single-ended-differential-amplifiers/

\bibitem{b2} N. W. Emanatoglu. (2018) Task 4 [Online]

\bibitem{b3} N. W. Emanatoglu. (2018) Task 4 briefing [Online]
 

\end{thebibliography}

\end{document}
