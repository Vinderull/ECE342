\begin{document}

The design, simulation, and construction of experiments to measure the performance of a differential amplifier is explored.Two differential amplifiers were tested. The first amplifier is a resistively loaded amplifier, while the second differential amplifier has an active load. The common mode rejection ratio, input common mode voltage range and differential gain verses singled-ended gain were measured. It was found that the active loaded differential amplifier was more efficient with less passive components, and also produced a higher gain than the resistively loaded differential amplifier. The double-ended resistively loaded with cascode current mirror amplifier has a differential gain of 19 dB with a common mode gain of approximately -55 dB and a CMRR max of 90 dB at 1V. The double-ended active loaded amplifier has a max differential gain of 30 dB has a max common mode gain of approximately -29 and a max CMRR of 59 dB. An important lesson learned was about the tradeoff between an optimal operation range and the final gain achievable by a circuit. In addition the benefits of both a simple and cascode current mirror were explored.

\end{document}