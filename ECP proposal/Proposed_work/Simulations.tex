


\begin{document}
This section describes the simulation of the MFBP using NGSpice integrated with Matlab.The frequency response, including center frequency, bandwidth, and gain were simulated.

The final circuit that was simulated is shown in Figure \ref{fig:simcircuit}. 
\begin{figure}[H]
    \centering
    \includegraphics[scale = .30]{Simulations/Lab_2_simulated.png}
    \caption{Simulated circuit of MFBP}
    \label{fig:simcircuit}
\end{figure}

\noindent Figure \ref{fig:simcircuit} uses the same design structure as Figure \ref{fig:schem} seen in Section 2. 
All of the simulated circuit values are summarized in Table \ref{tab:simcomp}.


\begin{table}[H]
\centering
\caption{Simulated values}
\label{tab:simcomp}
\begin{tabular}{|l|l|}
\hline
Component     & Value        \\ \hline
R$_1$         & 7k$\Omega$   \\ \hline
R$_2$         & 2.2k$\Omega$ \\ \hline
R$_3$         & 170k$\Omega$ \\ \hline
C$_1$ = C$_2$ & 470pF        \\ \hline
V$_{CC}$        & 12V          \\ \hline
V$_{EE}$        & -12V         \\ \hline
\end{tabular}
\end{table}

Table \ref{tab:simcomp} was simulated using the same values that were calculated in Section 2. These simulations were close to the required specifications.
\newline 


%%%%% TALK ABOUT GAIN, BANDWIDTH, AND CENTER FREQUENCY

The frequency response was simulated using an AC sweep in NGSpice. Figure \ref{fig:sim_freqresp} below shows the frequency response. The simulation was performed from 100Hz to 1MHz.
\begin{figure}[H]
    \centering
    \includegraphics[scale = .3]{Simulations/sim_frequency_gain_with_legend.jpg}
    \caption{Simulated AC analysis of MFBP}
    \label{fig:sim_freqresp}
\end{figure}

The center frequency in Figure \ref{fig:sim_freqresp} was found to be a little less than the required 20kHz. However, upon taking into account the frequency bandwidth, the MFBP is within specifications. The resistors available for design have a $\pm5\%$ tolerance about their nominal value. The available capacitor values have a$\pm10\%$ value about their nominal value. The effect of these tolerances on the performance is summarized in Figure \ref{fig:twosigma}. The simulation was performed in 10 iterations using Matlab random number generator to choose values that fell within the respective component tolerances[3].

\begin{figure}[H]
    \centering
    \includegraphics[scale = .3]{Simulations/resp_two_sigma_variation.jpg}
    \caption{Two-sigma variation}
    \label{fig:twosigma}
\end{figure}

As Figure \ref{fig:twosigma} the tolerances could lower the center frequency quite significantly. Care should be taken when constructing the circuit to choose components that fall close to their nominal values. Table \ref{tab:simvalue} summarizes the simulated results of the circuit.

\begin{table}[H]
\centering
\caption{Simulated results}
\label{tab:simvalue}
\begin{tabular}{|l|l|}
\hline
Value            & Simulated \\ \hline
Center Frequency & 19.45 kHz \\ \hline
Gain             & 65.1 dB   \\ \hline
Bandwidth        & 1.89 kHz  \\ \hline
\end{tabular}
\end{table}

The simulated results in Table \ref{tab:simvalue} fell within an acceptable margin of error for this simulation. 
    
    


\end{document}