\begin{document}
The final circuit is shown in Figure \ref{fig:finalexperimentalschem}.

\begin{figure}
	\centering
	\includegraphics[width=0.7\linewidth]{ExperimentalImplementation/NMOS_exp}
	\caption[NMOS final circuit schematic]{}
	\caption{}
	\label{fig:nmosexp}
\end{figure}


\begin{figure}
	\centering
	\includegraphics[width=0.7\linewidth]{ExperimentalImplementation/BJT_Exp}
	\caption[BJT final schematic]{}
	\caption{}
	\label{fig:bjtexp}
\end{figure}


\begin{table}[]
	\centering
	\caption{BJT experimental values}
	\label{my-label}
	\begin{tabular}{cc}
		$Q_1, Q_2, Q_3$ & 2N3904        \\
		$R_ref$         & 19.3k$\Omega$ \\
		$R_C$           & 4.7k$\Omega$  \\
		$R_B$           & 10k$\Omega$   \\
		$R_E$           & 3.3k$\Omega$  \\
		$R_L$           & 1k$\Omega$    \\
		$C_B, C_C, C_E$ & 470nF        
	\end{tabular}
\end{table}

In order to meet specifications, several minor changes to all circuits were made. The individual changes are discussed in their respective subsection as follows, the signal conditioner then the current driver.


%\subsection{Ring oscillator}
%\subfile{ExperimentalImplementation/ringosc.tex}



\end{document}