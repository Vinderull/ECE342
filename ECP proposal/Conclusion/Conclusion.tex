\begin{document}
The design, simulation, and implementation of the LED driver have been explained. Lab specification required that the signal generator have a frequency of approximately 20kHz, a duty-cycle of approximately 50\%, and an amplitude of at least 100mA. The LED driver takes a sinuisoidal waveform of variable duty-cycle and outputs a 50\% durt-cycle square wave. The waveform is then converted to a suffeciently large driving current by the current driver. The LED driver was constructed using the following parts: a 10k$\Omega$, 100k$\Omega$, 470$\Omega$, 1k$\Omega$, 12$\Omega$ and a 10k$\Omega$ potentiometer; a MCP6004 quadrature operation amplifier; an IR1503 LED; and finally a 2N3904 BJT. A 9V battery supply is stepped down to 5V with an LM7805 voltage regulator. The frequency was 20.1kHz, with a duty-cycle of 50.1\%, and an amplitude of 150mA. 
%%IMPORTANT LESSON
An important lesson about the behavior circuits including both op amps and transistors was learned. The meshing of op amps and transistors provide novel solutions to real world problems



\end{document}