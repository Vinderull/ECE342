\begin{document}
After several minor changes, the circuit operated correctly. This lab served as introduction to circuits that include both transistors and op amps. The two types of components have been studied separately, but never in conjunction. The specifications are outlined in Table \ref{tab:specs}.

\begin{table}[H]
	\centering
	\caption{Driver specifications}
	\label{tab:specs}
	\begin{tabular}{|l|l|}
		\hline
		Specifications & Required       \\ \hline
		Frequency      & 20kHz$\pm 5\%$ \\ \hline
		Duty-cycle     & 50\%           \\ \hline
		Amplitude      & $\geq$100mA    \\ \hline
	\end{tabular}
\end{table}

Notably, the ring oscillator had to have the its output frequency increased. This was due, in part, because of parasitic inductances and capacitances from the board and jumper wires. The propagation of the signal from the oscillator to output resulted in a decrease of frequency. Therefore, increasing the input frequency resulted in a correct output of the LED driver.

Another factor that determined the operating frequency is the fact that most of the components are temperature dependent. The MCP6004 IC, with increasing temperature, has an increased frequency output. The voltage regulator's efficiency, however, decreases with increasing temperature. The current through the LED is also a function of temperature. The behavior of this circuit can be heavily dependent on ambient temperature. This is a vital stipulation as, depending on the bandwidth of the MFBP filter designed before, the LED signal may fall inside the stopband of the MFBP filter. The receiver would then fail to receive the LED signal and the optical uplink would not operate. The summary of the final results can be seen in Table \ref{tab:simvsexp}.


\begin{table}[H]
	\centering
	\caption{Simulated vs. experimental results}
	\label{tab:simvsexp}
	\begin{tabular}{|l|l|l|}
		\hline
		Component Values  & Simulated & Experimental \\ \hline
		Output Current    & 200mA     & 150mA        \\ \hline
		Output Frequency  & 20kHz     & 20.1kHz      \\ \hline
		Output Duty-cycle & 48\%      & 50.1\%       \\ \hline
	\end{tabular}
\end{table}

The final circuit fell well within specifications and operated correctly after several component alterations.



\end{document}