\begin{document}

This lab served as introduction to MOSFETS and implementation of digital logic. When using transistors to create digital devices, it is often assumed that the device behaves ideally. When the device is logic high, it is one voltage, and when it is logic low it is another. Based on this logic, it is assumed that the switching on and off of the MOSFETS happen instantaneously. Based on the results of the ring oscillator and other inverter circuits, the MOSFETS do not switch instantaneously. Instead, power is consumed by the MOSFETS when they transition from one state to another. This consumption of power causes a small time delay. Each of the circuits described in this report are discussed in the following subsections, excluding the astable multivibrator. The astable multivibrator was not constructed during the experimental phase of this task.


\begin{table}[H]
	\centering
	\caption{Simulated vs. experimental results}
	\label{tab:simvsexp}
	\begin{tabular}{|l|l|l|}
		\hline
		Component Values  & Simulated & Experimental \\ \hline
		Output Current    & 200mA     & 150mA        \\ \hline
		Output Frequency  & 20kHz     & 20.1kHz      \\ \hline
		Output Duty-cycle & 48\%      & 50.1\%       \\ \hline
	\end{tabular}
\end{table}


\subsection{Ring oscillator}
The ring oscillator circuit designed with the CMOS did not initially meet the specifications required in the lab. The simulations were accurate, but real-world parasitic capacitances from the board and jumper wires impacted the circuit. To remedy the situation, a capacitor of 4.7 nF was added in parallel after the second inverting gate. By increasing the capacitance, we lowered
the frequency from 22.8 kHz to 19.8 kHz. The comparison between the simulated results and experimental is expressed in Table \ref{tab:simvexpring}

\begin{table}[H]
	\centering
	\caption{Comparison of ring oscillator results}
	\label{tab:simvexpring}
	\begin{tabular}{|l|l|l|}
		\hline
		Component Values & Simulated & Experimental \\ \hline
		C$_1$            & 8nF       & 8.2nF        \\ \hline
		C$_2$            & 8nF       & 13.9nF       \\ \hline
		C$_3$            & 8nF       & 8.2nF        \\ \hline
		Frequency        & 20.1kHz   & 19.8kHz      \\ \hline
		Amplitude        & 5V        & 4.9V         \\ \hline
	\end{tabular}
	\end{table}
The reason this oscillator is chosen over the astable multivibrator as the signal generator is mostly arbitrary and simply design choice for the optical uplink project \cite{b4}. Although one of the benefits of the ring oscillator is that from the sensitivity testing it is less affected by component tolerances than the astable multivibrator. This is significant since the capacitors available for circuit construction are 10\% tolerances.


\end{document}