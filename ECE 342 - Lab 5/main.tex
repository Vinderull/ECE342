\documentclass{article}
\usepackage{fullpage} % sets more standardized margins
\usepackage{graphicx} % some graphics functions I use 
\usepackage{abstract} % abstract function
\usepackage{amsmath}  % math stuff
\usepackage{float}
\usepackage{mathrsfs}
\usepackage[utf8]{inputenc}
\usepackage[document]{ragged2e}
\usepackage{subfiles}
\usepackage{caption}
\usepackage{subcaption}
\usepackage{verbatim}

\renewcommand{\absnamepos}{flushleft} % left justifies abstract
\setlength{\absleftindent}{0pt}
\setlength{\absrightindent}{0pt}



% % %
% Set up IEEE style paragraphing
% % %
\setlength{\parskip}{1em} % The \par command now skips a line between paragraphs, eliminates warnings from using the \par or \\ commands
\setlength{\parindent}{0em} % Left justifies paragraphs after a \par command 


\begin{document}

\pagenumbering{gobble} % Turn off page numbering for titles and tables


\begin{titlepage}
    \begin{center}
        
         \vspace*{1.5cm}
        
    %Berkay suggested this title in the meeting
         \textbf{{\Huge Task 5: Voltage Amplifier }}
        
         \vspace{0.5cm}
         \textbf{Optical Uplink}
        
          \vspace{.5cm}
        
         \textbf{{\Large Joseph Arsenault \\ Ryan Dufour \\ Phil Robb \newline}}
    \vspace{1cm}
    
        
        
        \begin{abstract}
        
        
        %I re-worded it here, check it over to make sure it sounds not so choppy and details what is needed
        
        \noindent The design, simulation and construction of an amplifier circuit are described. In task 5 of the optical uplink project, a voltage amplifier will increase the signal from the active bandpass filter from a low voltage range, possibly in mV, to 5V. Two ways are explored to accomplish this task. The first is using a common source NMOS circuit. The second is using a NPN BJT common emitter circuit. For the purposes of the optical uplink project, the NMOS was chosen instead of the BJT. The cutoff frequency for the BJT was greater than 1kHz, while the NMOS had a cutoff frequency of approximately 600 Hz.
        
       % The design, simulation, and construction of an LED driver circuit are described. In task 4 of the optical uplink project, a signal conditioning circuit and current driver are explored. A signal conditioner using the MCP6004 operational amplifier as a Schmitt trigger was built in order to output a square wave with 50\% duty-cycle. The current driver was constructed using a MCP6004 operational amplifier which drives an 2N3904 BJT. The current driver receives the output voltage from the conditioner and converts it to a current signal which controls an IR1503 LED. The output of LED driver was required to operated at approximately 20 kHz, 50\% duty-cycle, with a current of at least 100mA amplitude.  The final current operated at a frequency of 20.1 kHz, a duty cycle of 50.1\% and a 150mA peak amplitude.
        
        

        \end{abstract}
        
        
        \vspace{02.5cm}
        \textbf{
        Electrical and Computer Engineering\\
        University of Maine\\
        ECE - 342\\\ \today}
    \vspace{.5cm}
        \begin{figure}[H]
        \centering
        \includegraphics[scale = 0.1]{Images/barcode.png}
\end{figure}
    \end{center}
\end{titlepage}


\tableofcontents

\newpage

\newpage
\listoffigures
\listoftables
\newpage 
\clearpage

\pagenumbering{arabic} % Turn on page numbering



\section{Introduction}
\subfile{Introduction/Introduction}
  
 
  \section{Circuit Development}
    \subfile{CircuitDevelopment/Circuit.tex}
  
    
  %\subfile{CircuitDevelopment/cd4007SIM.tex}

   % \subfile{CircuitDevelopment/NMOS.tex}
     
 
    
  \section{Experimental Implementation}
  	\subfile{ExperimentalImplementation/experimental.tex}

    
  \section{Discussion}
     \subfile{Discussion/Discussion.tex}
    
    
    \section{Conclusion}
        \subfile{Conclusion/Conclusion.tex}

    
    \newpage
\clearpage

\appendix

\begin{thebibliography}{00}

\bibitem{b1} D.E. Kotecki Lab.(2017) Lab $\#5$ Voltage Amplifier [Online]. Available: http://davidkotecki.com/ECE342/labs/ECE342\_2017\_Lab5.pdf
\newline

\bibitem{NMOS} ON Semiconductor. (2017) 2N7000 [Online]. Available: http://www.onsemi.com/PowerSolutions/supportDoc.do?type=models\&rpn=2N7000
\newline

\bibitem{LEDDATA} Everlight Electronics. (2017) 5mm Infrared LED. [Online]. Available:
https://media.digikey.com/PDF/Data\%20Sheets/Everlight\%20PDFs/9-IR1503.pdf
\newline

\bibitem{b4} ON Semiconductor. (2017) LM7805 [Online]. Available: 
https://www.sparkfun.com/datasheets/Components/LM7805.pdf
\newline

\end{thebibliography}





\begin{comment}
\section{References}

\noindent [1] N.W. Emanatoglu.(2017) Lab $\#1$; photodetector and transimpedence amplifier [Online]. Available: http://web.eece.maine.edu/~kotecki/ECE342/labs/ECE342_2017_Lab2.pdf
\newline

\noindent [2] D.E. Kotecki Lab.(2017) Resistively Loaded MOSFET Gate [Online]. Available: http://web.eece.maine.edu/~kotecki/ECE342/labs/ECE342_2017_Lab3_Inverter.pdf
\newline

\noindent [3] N.W. Emanatoglu.(2017) Lab $\#1$; photodetector and transimpedence amplifier [Online]. Available: http://web.eece.maine.edu/~kotecki/ECE342/labs/ECE342_2017_Lab2_MFBF.pdf
\newline

\noindent [4] Cree.(2014) 5-mm Blue and Green LED [Online]. Available http://www.cree.com/led-components/media/documents/C503B-BAS-BAN-BCS-BCN-GAS-GAN-GCS-GCN-1094.pdf
\newline

\noindent [5] Kingbright.(2017) WP7113SGC [Online]. Available: http://www.us.kingbright.com/images/catalog/SPEC/WP7113SGC.pdf
\newline

\noindent [6] D.E. Kotecki Lab.(2017) PIN Silicon Photodiode [Online]. Available: http://web.eece.maine.edu/~kotecki/ECE342/datasheets/OP993-999\_0.pdf
\newline

\noindent [7] Kingbright.(2017) WP7113SEC/J3 [Online]Available: http://www.kingbrightusa.com/images/catalog/SPEC/WP7113SEC-J3.pdf
\newline

\noindent [8] Everlight.(2006) IR1503 [Online]. Available: https://radiodetali.com/media/catalog/product/i/r/ir1503.pdf
\newline
\end{comment}

\end{document}
