\begin{document}

The design, simulation, and construction of experiments to measure the performance of an operational amplifier was explored. The amplifier consisted of three discrete stages, an actively loaded differential pair, a common source amplifier, and a BJT output stage. The differential gain was found to be 67 dB while the common mode rejection ratio was measured at 75 dB. This circuit will serve as the foundation for the final design of the operational amplifier in task 6. The operational amplifier still requires frequency compensation in order to maintain stability. An important lesson learned during this lab was that simply achieving a high gain is not enough for a circuit to be considered operational. Whether or not the op amp is stable is as important as the final gain value.
\end{document}