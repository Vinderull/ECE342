\begin{document}
After several minor changes, both circuits operated correctly. This lab served as introduction to the main differences between MOSFET and BJT devices. The two components are both transistors but have defining differences, mainly the polarity of the devices. The specifications are outlined in Table \ref{tab:specs}.

\begin{table}[H]
	\centering
	\caption{BJT and NMOS comparison}
	\label{tab:specs}
	\begin{tabular}{|c|c|c|c|}
		\hline
		BJT             & Values        & NMOS            & Values       \\ \hline
		$Q_1, Q_2, Q_3$ & 2N3904        & $N_1, N_2, N_3$ & 2N7000       \\ \hline
		$R_ref$         & 19.3k$\Omega$ & $R_Ref$         & 22k$\Omega$  \\ \hline
		$R_C$           & 4.7k$\Omega$  & $R_D$           & 8.2k$\Omega$ \\ \hline
		$R_B$           & 10k$\Omega$   & $R_G$           & 1M$\Omega$   \\ \hline
		$R_E$           & 3.3k$\Omega$  & $R_S$           & 8.2k$\Omega$ \\ \hline
		$R_L$           & 1k$\Omega$    & $R_L$           & 1k$\Omega$   \\ \hline
		$C_B$, $C_C$, $C_E$ & 470nF     & $C_G$, $C_S$, $C_D$ & 10$\mu$F     \\ \hline
		Gain            & 25 dB         & Gain            &  22 dB       \\ \hline
	\end{tabular}
\end{table}


Notably, different voltages than those specified in the lab manual were used as the Analog Discovery handles small voltage signals very poorly, producing a lot of noise.  Other than this both circuits had little issue performing as expected.

The BJT, while having a larger gain and much less 2nd harmonic distortion, did not have the required cutoff frequency to meet the specification for the optical link project, thus the NMOS CS amplifier was chosen for it's ability to meet specification.

One of the biggest challenges of this lab was the mathematical assumptions and operations used to solve for the transistor values in simulation. When dealing with such devices the assumptions used can sometimes give wildly incorrect values and lead to repetition of calculations until correct. Fortunately simulations help greatly with reducing the time it takes to get to the correct solution.

The final circuits fell well within specifications and operated correctly after minor component alterations.



\end{document}