\begin{document}
After several minor changes, the op amp operated correctly. This lab served as introduction to feedback applications of operational amplifeirs. The two types of feedback were explored. The specifications are outlined in Table \ref{tab:specs}.

\begin{table}[H]
	\centering
	\caption{Current and voltage values from simulated vs experimental circuit}
	\label{tab:specs}
	\begin{tabular}{|l|c|c|}
		\hline
		\multicolumn{3}{|l|}{Simulated current values}                            \\ \hline
		\textbf{DC Bias Conditions} & Simulated & Experimental        \\ \hline
		V$_{Ref_2}$ & -340 mV      & -403 mV                                        \\ \hline
		V$_{Ref_1}$ & -2.987 V       & -3.01 V                        	   \\ \hline
		V$_{D_1}$ & 2.796 V          & 2.82 V                              	       	       \\ \hline
		V$_{D_2}$  & 2.796 V                       & 2.84 V                        	   \\ \hline\
		V$_{out}$ & $\approx$ 0 V				& .09mV 						 \\ \hline
		I$_{Ref}$ & 410.8 $\mu$A            & 387 $\mu$A                                       \\ \hline
		I$_{D_1}$ & 204.4 $\mu$A              &  193.7 $\mu$A\                                    \\ \hline
		I$_{D_2}$ & 204.4 $\mu$A                  & 194 $\mu$A                                \\ \hline
		I$_{CS}$  & $\approx$  227 $\mu$A               & 217 $\mu$A                            \\ \hline
		I$_{C}$   & $\approx$ 2  $\mu$A				& 2.23 $\mu$A 						 \\	\hline

	\end{tabular}
\end{table}


One of the bigger challenges of this lab was achieving and maintaining stability of the op amp as well as getting the op amp to operate at all, the op amp was adjusted and rebuilt more than once until it was finally able to perform the experiments required for this lab. This challenge arises from the volatility of a discrete op amp as well as well as flaw inherent in the manufacturing of the devices.

The final circuits fell well within specifications and operated correctly after minor component alterations.



\end{document}