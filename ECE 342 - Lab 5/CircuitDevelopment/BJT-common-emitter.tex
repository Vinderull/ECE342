\begin{document}
The BJT operates similarly to the NMOS circuit. The difference being it is now an NPN transistor and is therefore a current controlled voltage source, as opposed to the NMOS which is a voltage controlled. The values for the BJT circuit were not solved for, but instead provided as part of the lab, which can be seen in Table \ref{tab:bjttab}

\begin{table}[H]
	\centering
	\caption{BJT experimental values}
	\label{tab:bjttab}
	\begin{tabular}{cc}
		$Q_1, Q_2, Q_3$ & 2N3904        \\
		$R_ref$         & 19.3k$\Omega$ \\
		$R_C$           & 4.7k$\Omega$  \\
		$R_B$           & 10k$\Omega$   \\
		$R_E$           & 3.3k$\Omega$  \\
		$R_L$           & 1k$\Omega$    \\
		$C_B, C_C, C_E$ & 500nF        
	\end{tabular}
\end{table}

 The simulations of the BJT follow and were performed in Matlab integrated with NGspice. The frequency response of the circuit can be seen in Figure \ref{fig:bjtsimfreq}. 

\begin{figure}[H]
	\centering
	\includegraphics[width=.55\textwidth]{CircuitDevelopment/BJT_bandwidth.jpg}
	\caption{Simulated frequency response of BJT}
	\label{fig:bjtsimfreq}
\end{figure}

Notably, the 3dB lower cutoff is greater than 1 kHz. The corresponding passband gain is significantly higher than 20 dB. This is not ideal, because the output could begin to saturated due to excessive gain.



 



\end{document}