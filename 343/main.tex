\documentclass{article}
\usepackage{fullpage} % sets more standardized margins
\usepackage{graphicx} % some graphics functions I use 
\usepackage{abstract} % abstract function
\usepackage{amsmath}  % math stuff
\usepackage{float}
\usepackage{mathrsfs}
\usepackage[utf8]{inputenc}
\usepackage[document]{ragged2e}
\usepackage{subfiles}
\usepackage{caption}
\usepackage{subcaption}
\usepackage{verbatim}


\renewcommand{\absnamepos}{flushleft} % left justifies abstract
\setlength{\absleftindent}{0pt}
\setlength{\absrightindent}{0pt}
%\captionsetup[figure]{font=small,skip=0pt}


% % %
% Set up IEEE style paragraphing
% % %
\setlength{\parskip}{1em} % The \par command now skips a line between paragraphs, eliminates warnings from using the \par or \\ commands
\setlength{\parindent}{0em} % Left justifies paragraphs after a \par command 


\begin{document}

\pagenumbering{gobble} % Turn off page numbering for titles and tables


\begin{titlepage}
    \begin{center}
        
         \vspace*{1.5cm}
        
    %Berkay suggested this title in the meeting
         \textbf{{\Huge Task 4: Operational Amplifier}}
        
         \vspace{0.5cm}
         \textbf{Discrete Amplifier}
        
          \vspace{.5cm}
        
         \textbf{{\Large Joseph Arsenault \\ Ryan Dufour \\ Phil Robb \newline}}
    \vspace{1cm}
    
        
        
        \begin{abstract}
        
        
        %I re-worded it here, check it over to make sure it sounds not so choppy and details what is needed
        
        \noindent   The design, simulation, and construction of experiments to measure the performance of a differential amplifier is explored.Two differential amplifiers will be tested. The first amplifier is a resistively loaded amplifier, while the second differential amplifier has an active load. The common mode rejection ratio, input common mode voltage range and differential gain verses singled-ended gain will be measured. The double-ended resistively loaded with cascode current mirror amplifier has a differential gain of 19 dB with a common mode gain of approximately -55 dB and a CMRR max of 90 dB at 1V. The double-ended active loaded amplifier has a max differential gain of 30 dB has a max common mode gain of approximately -29 and a max CMRR of 59 dB. 

        \end{abstract}
        
        
        \vspace{03cm}
        \textbf{
        Electrical and Computer Engineering\\
        University of Maine\\
        ECE - 343\\\ \today}
    \vspace{.5cm}
        \begin{figure}[H]
        \centering
        \includegraphics[scale = 0.1]{Images/barcode.png}
\end{figure}
    \end{center}
\end{titlepage}


\tableofcontents

\newpage

\newpage
\listoffigures
\newpage 
\clearpage

\pagenumbering{arabic} % Turn on page numbering



\section{Introduction}
\subfile{Introduction/Introduction}
  
 
  \section{Circuit Development}
    \subfile{CircuitDevelopment/Circuit.tex}
  
  \section{Simulation}
    \subfile{Simulations/simulations.tex}
    
  %\subfile{CircuitDevelopment/cd4007SIM.tex}

   % \subfile{CircuitDevelopment/NMOS.tex}
     
 
    
  \section{Experimental Implementation}
    \subfile{ExperimentalImplementation/experimental.tex}
    
  \section{Discussion}
     \subfile{Discussion/Discussion.tex}
    
    
    \section{Conclusion}
        \subfile{Conclusion/Conclusion.tex}

    



\clearpage

\appendix

\begin{thebibliography}{00}

\bibitem{b1} All About Circuits. (2018) Single-ended differential amplifiers [Online]: https://www.allaboutcircuits.com/textbook/semiconductors/chpt-8/single-ended-differential-amplifiers/

\bibitem{b2} N. W. Emanatoglu. (2018) Task 3 [Online]


\end{thebibliography}

\end{document}
